\chapter*{Prefacio}
\addcontentsline{toc}{chapter}{Prefacio}

Hace tiempo entrevisté a una maestra para trabajar conmigo en una empresa donde damos cursos. Después de hablar sobre su experiencia, le hice una pregunta diferente:

``Si tuvieras todo el dinero del mundo y ya hubieras comprado todo lo que quisieras —carros, viajes, casa— y tus necesidades y gustos estuvieran cubiertos, ¿qué harías con tu vida? ¿Qué elegirías hacer si el dinero no fuera una preocupación ni un objetivo?''

Me dijo que haría exactamente lo mismo para lo que estaba a punto de contratarla: enseñar. Le gustaba aprender y le gustaba enseñar. Dijo que seguramente tomaría su tiempo libre para tomar cursos, diplomados, una maestría, prepararse más y luego seguir enseñando frente a un grupo. La contraté de inmediato. Hasta la fecha es una colaboradora excelente, porque no necesito pedirle que haga lo que naturalmente ya quiere hacer. Trabaja extra no porque se lo exija, sino porque lo disfruta. Hace las cosas con excelencia no para cumplir expectativas, sino porque ella busca hacerlo de esa manera. Yo solamente le ofrecí un luhar en donde ella pudiera ser ella misma.

Esa conversación me hizo pensar en cuántas personas viven exactamente lo contrario.

\vspace{0.5cm}

En diciembre del 2025, un grupo de amigos y yo estábamos escribiendo nuestros propósitos de año nuevo. Decidimos preguntarnos qué nos había estorbado para lograr nuestros sueños y metas durante ese año. Todos contestaron prácticamente lo mismo: su trabajo. Decían que su trabajo era lo que les estorbaba, lo que les quitaba tiempo, energía, lo que no les permitía enfocarse en algo que realmente les interesara.

Yo fui el último en responder. Dije que nada. Se burlaron de mí. Pero era cierto: nada me había estorbado para lograr lo que quería lograr. Tenía muy bien definido qué quería hacer, y a diferencia de ellos, yo amaba mi trabajo. Mi trabajo era parte del plan para lograr mis metas. No era una actividad más, sino algo que intencionalmente quería hacer todos los días.

El contraste me dejó pensando. ¿Por qué para mis amigos el trabajo era un obstáculo y para mí era el vehículo?

\vspace{0.5cm}

Me di cuenta de que hay mucha gente para la que el trabajo es una pausa en su día. El tiempo que le dedican no es suyo, es de alguien más. Como si solo pudieran empezar a disfrutar su vida una vez que llega la hora de la salida. De ocho a cinco es solamente cumplir con expectativas, tareas, no quedar mal, y esperar la salida para por fin poder continuar con sus vidas. Encuentran alivio los viernes y alargan el domingo lo más que se pueda para evitar el tan indeseado lunes.

Mucha gente vive así. Y qué triste es pensar que su energía y creatividad estén enfocadas en un trabajo al que no le encuentran ningún sentido.

\vspace{0.5cm}

Hubo un tiempo yo me sentí así, a inicios de mis veintes. Pero cuando por fin encontré lo que quería para mi vida, donde podía sentirme valioso y usar mis fortalezas, todo cambió. Cada uno de mis empleo tenía un propósito, y lo hacía con muchas ganas y con mucho amor. Recuerdo que cuando empecé a trabajar como docente me sentía realizado. Sabía lo que estaba buscando, y ese trabajo era parte del proceso.

Cuando una persona hace su trabajo solo porque sabe hacerlo, porque estudió cierta carrera, porque no tiene otra opción o porque no conoce nada más, pero no encuentra un propósito en ella, tarde o temprano se siente vacía. Hacer cosas sin propósito nos deja un vacío por dentro que ningún sueldo o empresa puedes llenar.

\vspace{0.5cm}

\textbf{Es por eso que decidí escribir este libro.}

Gracias a Dios, tuve la oportunidad de tomar riesgos en mis veintes y descubrir lo que realmente me apasionaba. Pero conozco muchos adultos que no están haciendo algo que disfruten, que no encuentran sentido en lo que hacen. He visto a personas cambiar de trabajo creyendo que una mejor empresa, un mejor sueldo o un jefe diferente resolverá su problema, solo para encontrarse a los pocos meses con el mismo vacío que no les asechaba. Porque el problema no estaba en la empresa. Estaba en que nunca encontraron su propósito, no conocen sus fortalezas y se sienten desconectados de lo que hacen.

Para mí, el trabajo es de las mejores cosas que nos pueden suceder como seres humanos. Nos sentimos útiles. Nuestra autoestima sube. Podemos ayudar a otras personas. Transformamos nuestras habilidades en un beneficio económico y social. Pero desgraciadamente, hay mucha gente que ve el trabajo como una condena, un lugar donde renuncian a sí mismos en cierto horario a cambio de una ilusión de seguridad.

La orientación vocacional está muy subestimada. No podemos separar el trabajo de nuestra vida personal. Si nos sentimos mal en lo personal, eso impacta en nuestro trabajo. Y si nuestro trabajo no tiene sentido, impacta en lo personal. Es imposible separarlos.

En muchas familias hay un dicho de que ``en la mesa no se habla de trabajo''. ¿Quién dijo eso? A mí me emociona mi trabajo, me encanta hablar de él, compartir lo que estamos logrando. ¿Por qué privamos a nuestros hijos de hablar de algo que será tan importante en sus vidas? Yo sé por qué: porque la mayoría de las personas, cuando hablan del trabajo, solo se quejan. Pero no tiene que ser así. En mi casa quiero hablar del trabajo para que mis hijos vean lo mucho que se puede ayudar a la gente y lo mucho que se puede lograr a través de él.

El trabajo es una expresión de nosotros mismos. \textit{No somos nuestro trabajo, pero en nuestro trabajo otros pueden ver lo que somos.}

\vspace{0.5cm}

No soy psicólogo ni orientador vocacional de profesión. Soy un maestro de matemáticas que disfruta tanto lo que hace, que ha podido crecer y cosechar frutos de su trabajo, y que intencionalmente ha construido su vida hacia donde sabe que encontrará más plenitud. En ese proceso he aprendido cosas que me gustaría compartir, porque sé que si otros las supieran y se dieran la oportunidad de experimentarlas, podrían sentirse muchísimo más plenos con su vida.

Y si eso es cierto para los adultos, imagina lo que puede significar para un adolescente que todavía tiene tiempo de elegir bien desde el inicio.

\vspace{0.8cm}

\textbf{Este libro lo escribo para los padres de familia de México y Latinoamérica.}

No solo para darles herramientas y conocimiento práctico que ayuden a la formación de sus hijos, sino también para inspirarlos a buscar plenitud en su propia actividad laboral, a encontrar sentido en su trabajo, a ser valientes para cambiar de rumbo y empezar desde cero si es necesario. Porque podemos decirle muchas cosas a nuestros hijos, pero el ejemplo arrasa. \textit{El ejemplo de buscar nuestros sueños es lo que hará que nuestros hijos quieran buscar los suyos.}

\vspace{0.8cm}

\textbf{¿Qué encontrarás en este libro?}

El libro está organizado en tres partes que siguen un orden lógico: primero los fundamentos, luego las herramientas prácticas, y finalmente el acompañamiento continuo.

\vspace{0.3cm}

\textbf{Parte I: Lo que todo padre debe saber}

En esta primera parte vamos a desmontar algunos mitos que probablemente cargas sin saberlo. Empezamos con una verdad incómoda: \textit{la carrera no define el éxito de tu hijo; la persona que es lo define}. Vas a conocer mi propia historia de cómo pasé de ser un ingeniero frustrado a un maestro realizado, y por qué el lugar donde trabajas importa más de lo que crees.

También vas a aprender los tres principios que, según la investigación, predicen el éxito mejor que cualquier título universitario. Y te daré algunos consejos para descubrir cómo identificar los talentos únicos de tu hijos.

\vspace{0.3cm}

\textbf{Parte II: De la teoría a la acción}

Aquí pasamos de entender a actuar. Vas a aprender por qué ``échale ganas'' no es un plan, y qué es lo que realmente determina si a tu hijo le irá bien o no en la vida profesional. Vamos a hablar de cómo funcionan las carreras en el mundo real y cómo conectar las fortalezas de tu hijo con las necesidades del mercado.

También te voy a dar un proceso paso a paso para elegir carrera para que tu hijo no elija desde la fantasía sino desde la evidencia. Y hablaremos un poco sobre los tests vocacionales: qué pueden hacer, qué no pueden hacer, y cómo usarlos correctamente.

\vspace{0.3cm}

\textbf{Parte III: El acompañamiento}

Esta es la parte que muchos libros de orientación vocacional ignoran: qué hacer \textit{después} de que tu hijo elige. Porque las cosas no siempre salen como uno espera.

Vamos a hablar de los escenarios difíciles —cuando quiere dejar la carrera, cuando termina pero ejerce algo diferente, cuando vive insatisfecho con lo que eligió— y cómo manejar cada uno sin destruir la relación ni el futuro de tu hijo. También vamos acerca de tomarse un año libre antes de entrar a la universidad, ya sea voluntaria o involuntariamente, cuándo tiene sentido, cuándo no, y cómo estructurarlo para que sea una inversión de tiempo y no una pérdida.

\vspace{0.3cm}

Cada capítulo termina con un reto práctico. No es un libro para leer y guardar en el librero. Es un libro para leer, reflexionar y actuar.

\vspace{0.8cm}

\textbf{¿Cómo leer este libro?}

Te recomiendo leerlo completo al menos una vez, aunque tu hijo esté en diferentes etapas. Los principios de la Parte I aplican desde la infancia. Las herramientas de la Parte II son más útiles cuando tu hijo está en secundaria o preparatoria. Y la Parte III te prepara para lo que viene después.

Si tu hijo está en primaria o secundaria, léelo con calma. Tienes tiempo para ir implementando cada capítulo. Si tu hijo ya está en preparatoria o a punto de elegir, necesitarás tomar acción más rápido —pero aun así no te saltes la Parte I, porque es muy importante.

Mi única petición es esta: no leas pasivamente. Cada vez que algo te haga pensar diferente sobre ti, sobre tu hijo, o sobre la relación entre ustedes, detente. Reflexiona. Y si es necesario, actúa.

Por ejemplo, si después de leer algo entiendes mejor a tu hijo en algo que antes no comprendías, acércate y dile: ``Hijo, perdóname, no había entendido esto que me decías. Ahora me gustaría entenderlo, platícame más''. Esa conversación vale más que cien páginas leídas sin acción.

Tu hijo necesita tu presencia y tus palabras. Aunque lo veas grande, todavía necesita escuchar ciertas cosas de ti, ver ciertos ejemplos, ciertas acciones. Necesita que seas un papá o una mamá valiente.

\vspace{0.8cm}

Tampoco me considero escritor. Me gusta leer y disfruto mucho aprender y ,unque comunicarme por escrito nunca ha sido mi fuerte, las palabras que están plasmadas en este libro sé que tienen el poder de marcar una diferencia en muchas familias. Hay mucho tiempo, intención, experiencia y amor en cada una de estas páginas.

\vspace{1cm}

\begin{center}
\rule{0.5\textwidth}{0.4pt}
\end{center}

\vspace{0.5cm}

No escribí este libro para darte todas las respuestas. Lo escribí para ayudarte a hacer las preguntas correctas.

La decisión final siempre será de tu hijo. Pero la preparación que le des —el carácter que formes, los talentos que descubras, las experiencias que le ofrezcas, el ejemplo que le muestres— eso está en tus manos.

El momento de actuar no es cuando tu hijo llegue a sexto semestre de preparatoria sin saber qué estudiar. El momento de actuar es ahora.

Espero que este libro te dé las herramientas para hacerlo.

Si este libro te es de ayuda, te pido un favor. Pásalo. Recomiéndalo. Compártelo con otros padres que lo necesiten. Porque lo valioso no se esconde, se comparte.

\vspace{1.5cm}

\begin{flushright}
\textit{Gastón Santos G.}\\
\textit{Tijuana, México}\\
\textit{Enero, 2026}
\end{flushright}