% Capítulo 1
\chapter{La persona define el destino, no la carrera}

\section{El lugar sí importa}

A los 26 años, después de algunos años trabajando como ingeniero industrial, me senté frente a mi madre y le dije algo que sabía que no le iba a gustar mucho escuchar.

``Ma, quiero ser maestro, quiero llegar a ser director y algún día tener mi propia escuela.''

Se quedó callada. Luego me miró con cierta preocupación que solo una madre puede expresar. ``Nunca quise eso para ti, hijo''. Yo ya lo sabía. Toda mi vida me había dicho lo mismo. ``No seas maestro.''

Ella había pasado décadas dando clases de matemáticas. Era excelente. Sus alumnos la adoraban y sus directores la valoraban muchísimo. Pero ella sabía que, aunque la vida de docente era una vida cómoda, venía acompañada de ciertas limitaciones financieras, no se podía vivir de una forma holgada, con lujos. Ella no quería eso para mí. Quería que yo tuviera más opciones, más oportunidades, más caminos abiertos. Quería que las cosas que fueron difíciles para ella fueran más fáciles para mí.

Yo había hecho exactamente lo que ella me pidió. Estudié Ingeniería Industrial. Busqué trabajo en las maquilas. Perseguí el sueño que ella tenía para mí. Intenté ser ese gran ingeniero que ella siempre había querido que yo fuera.

Pero ya no podía seguir así. Le expliqué por qué.

``Ma, pasé años intentando ser el ingeniero que tú querías y no logré sentirme yo mismo. No porque sea incapaz, sino porque ese lugar me apagó. Necesito intentar esto. Dame una oportunidad de demostrarte que puedo lograr lo que tú quieres haciendo lo que yo realmente quiero hacer.''

Ella guardó silencio. Luego dijo algo que nunca olvidaré.

``Si quieres ser director de tu propia escuela, necesitas mucho dinero. Mejor ponte a trabajar, ahorra mucho, y luego piensas en eso. Pero trabajando de maestro va a estar muy difícil.''

Ahí lo vi claro. Mi mamá no estaba tratando de controlarme. Estaba tratando de darme lo que ella sintió que le faltó. Oportunidades. Crecimiento. Mi madre no estaba siendo pesimista. Simplemente estaba siendo realista. Ella sabía que los sueños grandes requieren recursos grandes. Y desde su experiencia, un salario de maestro no iba a ser suficiente para llegar ahí.

Pero ese es el peligro cuando orientas a tus hijos desde tus propias limitaciones y malas experiencias, no desde su potencial y sus fortalezas.

Mis padres son docentes de matemáticas ya jubilados —toda su vida dedicada al aula. Crecí rodeado de calendarios escolares, exámenes apilados en la mesa del comedor, y conversaciones sobre estudiantes durante la cena. El ambiente escolar era mi normalidad. Pero intenté escapar de él.

Pasé años intentando encajar en un lugar que no era para mí. La maquila se me hizo fría. El ambiente de la maquila me resultaba ajeno. No importaba cuánto me esforzara, no lograba destacar. Me sentía frustrado, con baja autoestima, lleno de desconfianza en mí mismo. La maquila te aniquila, como dicen por ahí.

Las preguntas difíciles empezaron a aparecer. \textit{¿Por qué, si soy trabajador e inteligente, no puedo destacar como ingeniero?}

Ahora lo entiendo. Un trabajo que no encaja con quién eres produce desequilibrio en todas las partes de tu vida. La insatisfacción laboral no se queda en la oficina —te persigue a casa, te roba el sueño, te apaga la alegría, te quita la motivación. Un trabajo tedioso, realizado en condiciones inadecuadas o sujeto a tensiones constantes, puede hartarte fácilmente y hasta llegar a enfermarte. En cambio, una actividad interesante, gratificante y con propósito puede compensar muchas deficiencias, darte propósito, sentido, hacer que llevar una vida llena de alegría y satisfacción sea mucho más sencillo. Vivir haciendo lo que realmente quieres hacer provoca que te sientas feliz cuando llega el lunes. 

\vspace{0.5cm}

La conversación con mi papá fue completamente diferente.

Mi papá, con su estilo más exploratorio, me dijo algo que cambió todo.

\begin{quote}
Mira, todos los directores que yo conozco empezaron siendo maestros. ¿Qué te parece si das unas horas de clase y ves cómo te sientes? Pruébalo. Si no funciona, buscas otra cosa. Pero al menos sabrás.
\end{quote}

Fue el mejor consejo que pude haber recibido. No me dijo qué hacer. Me dio permiso para explorar.

Una semana después, mi papá, usando algunos contactos, logró conseguirme 20 horas de matemáticas en una escuela privada en Tijuana. Primer y quinto semestre. Iba a empezar el lunes siguiente.

Recuerdo el domingo en la noche, pensando: \textit{¿Y si mi mamá tiene razón? ¿Y si esto es un error?}

\vspace{0.5cm}

El primer día llegué 30 minutos antes. Tenía la planeación de clase en una carpeta azul que había comprado especialmente para la ocasión. Entré al salón vacío, escribí mi nombre en el pizarrón y esperé.

Los nervios estaban ahí, pero no eran de miedo. Eran de esos nervios buenos, como cuando sabes que algo importante está por comenzar. Había una certeza extraña. Una sensación de estar exactamente donde debía estar.

Cuando entraron los alumnos de primer semestre, esa certeza se confirmó.

Lo sorprendente fue la velocidad con la que me adapté. En solo unos días ya dominaba el ritmo del salón. Me adapté a levantar mi voz para que el estudiante de hasta atrás me pudiera escuchar, pero sin gritar para no asustar a los de enfrente. Me acostumbré rápidamente a los residuos de plumón en mis manos, a las carpetas llenas de listas de asistencias y exámenes  que me llevaba a casa para calificar, y  a que todos me saludaran en los pasillos mientras iba caminando de un salón a otro.

\textbf{Nunca, en los años que fui ingeniero, sentí eso. Nunca.}

En la maquila, cada día era una batalla contra mí mismo. Aquí, cada día era una oportunidad. Disfrutaba preparar mis clases —pensar cómo explicar ecuaciones de forma que hasta el alumno más distraído pudiera entenderlas. Me encantaba entrar al salón sabiendo exactamente qué iba a hacer, dar la clase de 50 minutos, y luego perfeccionar la misma explicación en la siguiente hora con otro grupo.

Los alumnos respondían. Me hacían preguntas. Se quedaban después de clase. Me buscaban en los pasillos para pedirme que les explicara algo que no habían entendido con otro maestro.

Después de cuatro meses frente al grupo, llamé a mi papá y le dije algo que me costó admitir: ``Creo que soy mejor maestro que ingeniero. No sé cómo explicarlo, pero con solo algunos meses dando clases me siento el mejor maestro de matemáticas de toda la escuela.''

\vspace{0.5cm}

De esa experiencia nació una verdad que ahora comparto en todas mis conferencias sobre orientación vocacional.

\textbf{El lugar sí importa.}

Hay un dicho popular que dice ``el que es cotorro donde sea es verde.'' Pero eso no siempre es verdad. Para empezar, porque nosotros no somos cotorros. Y segundo, porque ser bueno en algo específico no garantiza el éxito en cualquier carrera relacionada.

Yo era bueno con los números. Siempre lo fui. Pero eso no significaba que como ingeniero iba a destacar. Una habilidad aislada no define toda una carrera. Hay más factores que considerar. El ambiente. El propósito. El tipo de problemas que resuelves. Las personas con las que trabajas. La cultura del lugar.

Puedes ser inteligente, trabajador, disciplinado, y aun así no prosperar si estás en el lugar equivocado. No porque seas incapaz. Sino porque tus fortalezas no son valoradas ahí, porque el ambiente no te motiva, porque no le encuentras sentido a lo que haces.

Yo era la misma persona en la maquila y en el salón de clases. Misma inteligencia. Mismo esfuerzo. Pero en la maquila me sentía mediocre. En el salón me sentía extraordinario.

Cuando migré mi vida profesional al área correcta, todo cambió.
\begin{itemize}
\item Mi autoestima subió
\item La confianza en mis planes subió
\item Los elogios llegaron naturalmente
\item Pude ver un futuro claro para mí
\end{itemize}

\textbf{El lugar sí importa. La carrera sí importa. Lo que haces importa y cómo te sientes al respecto importa muchísimo.}

Puedo decir que gran parte de lo que he podido lograr fue porque con apoyo de mis padres pude cambiar de profesión. Me atreví a empezar de nuevo a mis 26, sin miedo. Y valió totalmente la pena.

Porque cuando se trata de hacer lo que realmente tienes que hacer, \textbf{el tiempo perfecto es HOY.}

\section{La ansiedad del sexto semestre}

Terminé mi primer semestre como maestro con buenas evaluaciones de mis alumnos y una sensación de logro que nunca había experimentado. Pasaron las vacaciones de invierno. Regresamos a clases a mediados de enero.

Mis alumnos de quinto semestre ahora pasaban a sexto. El último semestre de preparatoria. El semestre donde todo está a punto de cambiar.

Lo noté de inmediato, desde el primer día de clases. El ambiente era distinto. Ya no había ese aire despreocupado de meses anteriores. Entré al salón y en lugar de las pláticas habituales sobre el fin de semana, escuché fragmentos de conversaciones tensas.

``¿Ya decidiste qué universidad?''

``Mi mamá quiere que estudie Medicina, pero yo no sé...''

``El examen de admisión a la universidad está imposible.''

``Si no entro, mis papás me van a matar.''

El ambiente había cambiado radicalmente. Ya no hablaban de fiestas ni futbol. Hablaban de carreras, universidades, exámenes de admisión. Y no lo hacían con emoción. Lo hacían con ansiedad. Como si estuvieran a punto de tomar la decisión más importante de sus vidas sin tener idea de cómo tomarla.

Y tenían razón. Lo era. Y no tenían idea.

Yo los escuchaba con atención. Me gustaba hacer preguntas. Entender cómo estaban procesando esta decisión gigante. Lo que descubrí me alarmó. Las respuestas que escuché me dejaron helado.

\begin{itemize}
\item ``Voy a buscar en Google las carreras mejor pagadas y elijo la primera.''
\item ``Mis papás son abogados, yo voy a estudiar Derecho.''
\item ``Quiero Medicina porque dicen que pagan bien.''
\item ``Ingeniería porque ahí siempre hay trabajo.''
\end{itemize}

Ninguno hablaba de lo que \textit{disfrutaba} hacer. Todos hablaban de lo que \textit{creían} que era lo que tenían que hacer para ser ``exitosos''.

Dinero. Estatus. Aprobación familiar. Seguridad. Todo menos vocación.

Esa misma semana, después de clases, fui a una librería y me compré un libro llamado \textit{El profesor y la orientación vocacional}.\footnote{Cortada de Kohan, N. (2008). \textit{El profesor y la orientación vocacional} (2.ª ed.). Editorial Trillas.} Lo leí emocionado a diario. Subrayé párrafos enteros. Tomé notas. Quería ayudar a mis alumnos, pero no tenía las palabras ni los principios. Necesitaba prepararme. Sabía lo que se sentía cometer el error de elegir la carrera equivocada por las razones equivocadas.

\subsection{La niña de los plumones}

Recuerdo a una alumna en particular. Era la ``la niña de los plumones'' de su clase.

Siempre llegaba con una mochila repleta de marcadores Sharpie de todos los colores imaginables. Mientras yo explicaba derivadas en el pizarrón, ella tomaba apuntes que parecían páginas de un comic. Decoraba cada título con tipografías elaboradas, dibujos en los márgenes, esquemas visuales que convertían conceptos aburridos en arte.

Un día, al revisar su libreta, noté que había dibujado un pequeño cohete al lado de una ecuación. Le pregunté por qué.

``Es que cuando resuelvo problemas, me gusta imaginar que son misiones espaciales'', me dijo, un poco apenada. ``Hace que las matemáticas sean menos aburridas.''

Le pregunté si hacía eso en todas sus materias. Me mostró sus libretas de Historia, Química, Literatura. Cada una era una obra maestra visual. Me confesó que había descargado por su cuenta toda la suite de Adobe —Photoshop, Illustrator, InDesign— y que pasaba horas en YouTube viendo tutoriales de diseño.

Nadie le había pedido que lo hiciera. Lo hacía porque \textit{le encantaba}.

Y le hice la pregunta del millón: ``¿Ya decidiste qué vas a estudiar?''

``Derecho'', me contestó, sin mirarme a los ojos.

Me quedé callado un momento. Esperé. Luego pregunté con genuina curiosidad: ``¿Por qué Derecho?''

Suspiró. Como si hubiera estado esperando que alguien le hiciera esa pregunta.

``Mis papás tienen un despacho. Me dijeron que es la mejor idea. Cuando me gradúe voy a tener trabajo seguro en su oficina. Nunca voy a batallar. Y dicen que en Derecho pagan bien.''

Ahí estaba. El error clásico. Padres con buenas intenciones —padres que aman a su hija— tomando la decisión por ella, creyendo que ``trabajo seguro'' es sinónimo de éxito y felicidad. Sin preguntarle qué le apasiona. Sin observar lo obvio, que ella ya era diseñadora, solo le faltaba el título.

Me senté en el escritorio frente a ella. Le dije algo que había aprendido del libro que acababa de terminar.

\begin{quote}
Mira, lo que haces sin que nadie te lo pida es la señal más clara de tu vocación. Nadie te obliga a decorar tus libretas ni a ver tutoriales de diseño. Lo haces porque te encanta. Esa pasión natural vale más que cualquier promesa de trabajo seguro.
\end{quote}

Ella me miró, sorprendida. Como si nadie nunca le hubiera dicho eso.

Le expliqué algo más. Que eso la posicionaba muy por delante de cualquier diseñador promedio. Porque ella iba a hacer trabajo extra siempre. No porque su jefe se lo pidiera. Por puro gusto. Y quien hace más horas de práctica intencional, se vuelve mejor. Quien es mejor, eventualmente cobra más. Mucho más.

``Pero mis papás dicen que los diseñadores no ganan bien'', me interrumpió.

``Los diseñadores mediocres no ganan bien'', le respondí. ``Pero tú no vas por ese camino. Porque ya estás practicando. Ya tienes las herramientas. Ya estás construyendo un portafolio sin que nadie te lo pida. Eso te hace diferente. Todavía ni has empezado la universidad y ya tienes más experiencia que muchos que ya van a la mitad.''

Hice una pausa. Luego le dije algo más fuerte.

``Tener un trabajo seguro en el despacho de tus papás suena cómodo. Pero imagínate trabajar en algo que odias 40 o 50 horas a la semana durante 30 años. Levantarte cada lunes con la sensación de que estás desperdiciando tu vida. Eso no es seguridad. Es condena.''

Se quedó callada. Guardó sus plumones en la mochila lentamente. Al salir del salón, se volteó y me dijo: ``Gracias, profe.''

\vspace{0.5cm}

Años después me enteré que estudió Comunicación —una carrera que requiere habilidades creativas y sociales. Me la encontré en un Starbucks. Trabajaba como directora creativa en una agencia de publicidad. Me mostró su portafolio en su iPad. Ganaba el triple de lo que ganaban sus papás como abogados.

No sé si mi conversación influyó en su decisión. Pero me gusta pensar que sí.

Y me gusta más pensar que hay un padre leyendo esto ahora mismo que tiene una hija igual. Y que después de leer esta historia, tendrá una conversación diferente con ella.

\section{La investigación que me abrió los ojos}

Mientras daba clases en la preparatoria, seguía estudiando el MBA en Cetys Universidad en Tijuana. Clases los sábados. Proyectos entre semana. Todo porque seguía cargando esa visión heredada de mi mamá de querer ser ``algo más'' que solo un maestro.

Cuando llegó el momento de elegir tema para mi proyecto final de maestría, lo tuve claro de inmediato. Para terminar el MBA en Cetys, todos los estudiantes debíamos elaborar una tesina. A diferencia de una tesis doctoral que requiere investigación original y extensa, una tesina es un trabajo de investigación más breve y aplicado, enfocado en resolver un problema específico del mundo real.

Como era una maestría en administración de negocios, se esperaba que al final de la investigación propusiera un programa o modelo de negocio que resolviera el problema identificado. Pero lo que realmente me interesaba era el proceso de investigación. Los datos. Los patrones. Entender el \textit{por qué}.

Elegí investigar por qué los jóvenes desertan de la universidad. Tenía todo a la mano. La población objetivo en mi preparatoria, acceso directo a alumnos de sexto semestre a punto de elegir carrera, y una motivación personal profunda. Quería entender por qué tantos jóvenes entraban a la universidad con esperanza y salían años después con una carrera trunca y una deuda emocional gigante.

Diseñé encuestas. Hice entrevistas personales. Pasé meses recopilando datos. Y lo que encontré fue sorprendente.

En parte de mi investigación, encontré una estadística muy reveladora y preocupante.

Descubrí, según el INEGI, que del total de alumnos que logran entrar a la universidad, en promedio, solo el 30\% logra graduarse. Esta estadística es a nivel nacional, en universidades públicas del país, que son las que reciben a la mayor cantidad de adolescentes en México.

Es muy preocupante porque el proceso de admisión ya es extremadamente competido. Muchos jóvenes quedan fuera sin siquiera poder intentarlo. Y de los pocos que logran entrar, solo 3 de cada 10 terminan.

\textbf{Esto significa que 7 de cada 10 jóvenes que logra ingresar a la universidad terminan dejando su carrera trunca.} No por falta de inteligencia. No siempre por falta de dinero. Sino por una razón mucho más simple y fácil de solucionar.

Como maestro les puedo compartir que nuestro orgullo no solamente está en que el alumno pase su examen parcial del semestre, o que termine el semestre, ni siquiera que termine la prepa. Nuestro orgullo está en que logre entrar a la universidad y en poder encontrarnos en la calle años después y que nos digan: ``Hola profe, ¿qué tal? ¿Se acuerda de mí? Ahora soy doctor, ahora soy ingeniero, ahora soy docente.''

Ese es el verdadero orgullo que tenemos como maestros. Porque como dice Yoda, \textbf{``Somos lo que ellos llegan a ser. Esa es la verdadera carga del maestro.''}

\vspace{0.5cm}

Algo aún más sorprendente que la estadística de los pocos alumnos que llegan a graduarse de la universidad son las causas que llevan a esos alumnos a dejar la universidad.

Entre ellas se encontraban las que muchos nos imaginamos. Problemas financieros, la carrera se puso muy difícil, problemas de salud, cambios de ciudad. Pero esas causas representaban a la minoría de las personas que desertaban.

La causa número uno por la cual las personas dejan una carrera trunca es esta.

\textbf{``No era lo que yo esperaba.''} 

\textbf{``Siempre no me gustó.''}

Y esto es alarmante, no solamente porque representa a la mayoría de las personas que desertan de la carrera universitaria, sino porque \textbf{es una causa tan sencilla de arreglar y nadie está haciendo nada al respecto.}

Déjame repetirlo para que quede claro.

7 de cada 10 jóvenes abandonan la universidad porque eligieron una carrera que no era lo que esperaban o que terminó no gustándoles.

Esto no es un problema de dinero. No es un problema de capacidad intelectual. Es un problema de \textbf{orientación vocacional fallida.}

Y lo más frustrante es que este problema tiene solución. Pero requiere que tanto escuelas como familias hagan su trabajo. Y ahora mismo, casi nadie lo está haciendo.

\section{Las preparatorias olvidaron su propósito}

¿Para qué existe la preparatoria?

La respuesta debería ser obvia. Para, valga la redundancia, \textbf{preparar} al alumno para su educación superior, para la universidad. Para ayudarlo a descubrir sus fortalezas, explorar diferentes campos, y tomar una decisión informada sobre su futuro.

Pero si eso es el propósito, entonces tenemos un problema grave. Porque la mayoría de las preparatorias han olvidado completamente este propósito.

La orientación vocacional no debería ser una materia más en el programa de las preparatorias. Debería estar impregnada en todo lo que se hace ahí, porque ese es su verdadero propósito. De poco sirve el enfoque académico si solo formamos jóvenes inteligentes pero incapaces de identificar su lugar correcto en el mundo.

La orientación vocacional tendría que integrarse explícitamente en la escuela en todos sus niveles: talleres, experiencias reales, visitas a universidades y empresas, conversaciones con profesionales y exploración genuina. Mientras esto no suceda, seguiremos obteniendo el mismo resultado: jóvenes que eligen carreras de forma subjetiva, sin información real, guiados por estereotipos sociales y presiones familiares en lugar de por un auténtico autoconocimiento.

Pero la realidad es otra. Las preparatorias se han enfocado en cumplir programas académicos, no en orientar vidas.
\begin{itemize}
\item En el caso de las públicas, solamente quieren cumplir fechas, cumplir planes.
\item En el caso de las privadas, solamente quieren tener más inscripciones, más alumnos, más prestigio.
\end{itemize}

Hoy en día, muchas preparatorias ofrecen ``capacitaciones'' o ``especialidades'' que suenan impresionantes en el folleto de publicidad, pero cuando investigas qué están aprendiendo realmente los alumnos en esas capacitaciones, descubres que la mayoría están completamente desactualizadas.

Y lo peor no es que estén desactualizadas sino que estas capacitaciones no tienen el propósito de dejar al alumno explorar y descubrir si esa área le apasiona. Son simplemente una materia extra en el horario. Una cajita más que marcar en el plan académico.

Cuando hice encuestas en preparatorias públicas de mi ciudad, los números me sorprendieron.

\begin{itemize}
\item \textbf{30\% de los alumnos} que estaban en alguna capacitación o especialidad \textbf{ni siquiera querían estar ahí.} No eligieron esa especialidad. Les tocó por sorteo, por cupo lleno, por llegar tarde a la selección.
\item \textbf{90\% de las escuelas} ni siquiera te permiten cambiarte de especialidad o capacitación. Si te tocó Contabilidad y descubres que la odias en primer semestre, tienes que soportarla hasta sexto.
\end{itemize}

Piénsalo. Un joven de 15 años queda atrapado en una ``especialidad'' que no eligió, que no le gusta, y que se supone debería ayudarlo a orientarse vocacionalmente. Pero en lugar de orientarlo, lo está desorientando.

¿Cómo podemos decir con la cara seria que la preparatoria tiene el propósito de preparar a los alumnos para la universidad si ni siquiera se toma en serio el trabajo vocacional de mostrarle al alumno, por medio de experiencias reales, las diferentes áreas profesionales que existen?

\textbf{¿Cómo esperamos que un alumno escoja bien y llegue a sexto semestre con alguna idea clara de lo que quiere, si nunca le dimos la oportunidad de experimentar, comparar, equivocarse y ajustar?}

Este tema me molestaba profundamente. No podía quedarme callado viendo cómo año tras año, generación tras generación, jóvenes brillantes tomaban decisiones terribles por falta de información.

Me di a la tarea de preparar talleres ocasionales en mi ciudad. Empecé pequeño. Un taller de vez en cuando por mi cuenta. Luego me invitaron a otras escuelas. Eventualmente terminé dando conferencias en preparatorias públicas y privadas acerca de cómo elegir carrera. Abarco diferentes ejemplos, mitos, estrategias para que los alumnos puedan tomar esta decisión de manera informada.

Y es parte de lo que vamos a estar analizando en este libro. Estrategias que funcionan. Preguntas que importan. Ejercicios que revelan.

Pero hay algo más importante que cualquier estrategia o ejercicio.

\textbf{¿Cuál es el trabajo de papá y mamá en casa para poder guiar a nuestros adolescentes a un futuro próspero?}

Porque déjame decirte algo que tal vez no quieras escuchar.

\section{Los primeros responsables}

Si tu hijo está en sexto semestre y no tiene idea de qué estudiar, la escuela tiene parte de la culpa.

Pero tú tienes más.

Si bien la escuela de nivel medio superior tiene la responsabilidad de preparar a los alumnos para su educación universitaria, no son los únicos que tienen esta responsabilidad. Ni siquiera son los principales.

\textbf{Los primeros responsables de esta preparación son ustedes, padres de familia.}

No me malentiendas. No estoy diciendo que seas mal padre o mala madre. Estoy diciendo que el sistema educativo nos ha vendido una mentira gigante. Nos ha hecho creer que su trabajo es educar y el nuestro es solo proveer. Llevarlos a la escuela, comprarles útiles, pagar colegiaturas. Y confiar en que ``los expertos'' harán el resto.

Pero no es cierto.

Ustedes son los que tienen el rol más importante en la vida de sus hijos. No solamente por comprarles ropa y alimentarlos. Como diría Franco de Vita:.

\begin{quote}
No basta\\
Con llevarlos a la escuela a que aprendan\\
Porque la vida cada vez es más dura
\end{quote}

Desde la infancia hasta la adolescencia, como padres, son responsables de guiarlos, mentorearlos, exponerlos a diferentes áreas de la vida. Experiencias. Retos. Responsabilidades. Lugares donde ellos puedan no solamente conocer cómo funciona el mundo, sino algo más importante aún: \textbf{conocer cómo funcionan ellos mismos.}

¿En qué son buenos? ¿Qué disfrutan? ¿Qué los frustra? ¿Qué los emociona? ¿Qué despierta su interés?

Esas respuestas no se encuentran en un examen vocacional de 100 preguntas. Se encuentran en años de exposición intencional a diferentes experiencias. Y papá y mamá son quienes tienen la responsabilidad de crear esas oportunidades.

No es bueno decirles a los 17 años: ``Ya estás grande, ya tú sabrás qué carrera elegir.'' Porque aunque parezcan grandes, aunque tengan la estatura y la actitud de adultos, todavía no lo son. Todavía no conocen el mundo como tú lo conoces. Todavía no han cometido los errores que tú ya cometiste. Y si tú se los compartes de forma prudente —sin sermones, con historias— ellos pueden aprender sin tener que pasar por el mismo dolor.

\vspace{0.5cm}

Como profesor de matemáticas, me hablaban constantemente para dar asesorías personalizadas. Regularizaciones. Preparación para exámenes. Ayuda con tareas. La mayoría de las clases eran en las casas de los alumnos.

Esto me dio algo que pocos maestros tienen. Acceso.

Pude entrar a casas muy grandes en fraccionamientos privados con seguridad en la entrada. Y a casas muy pequeñas en colonias populares donde el techo de lámina apenas protegía de la lluvia. Familias muy ricas que me pagaban 500 pesos la hora. Y familias muy humildes que juntaban con esfuerzo para pagarme 200.

Y en cada casa, observé. Cómo hablaban los padres con sus hijos. Qué había en las paredes. Cómo pasaban su tiempo libre. Qué expectativas tenían. Qué hábitos practicaban.

Lo que vi me enseñó más sobre orientación vocacional que cualquier libro.

\subsection{El patrón de las familias estables}

Una de las cosas que pude observar de las familias que tenían más control sobre lo que estaban haciendo —su dinámica familiar, su estabilidad económica, su claridad de propósito— fue un patrón consistente.

\textbf{Siempre tenían a sus hijos en alguna clase de algo.}

Desde los 5 o 6 años. Inglés los martes y jueves. Karate los lunes y miércoles. Piano los sábados. Natación los domingos. No necesariamente todo al mismo tiempo, pero siempre algo.

Cuando les preguntaba a estos alumnos qué querían estudiar, casi siempre tenían una respuesta clara. O al menos dos o tres opciones bien pensadas. Habían probado cosas. Sabían qué se les daba bien y qué no. Tenían referencias.

``Profe, tomé clases de dibujo desde los 8 años. Me encanta. Voy a estudiar Arquitectura.''

``Profe, jugué futbol 10 años. No voy a ser profesional, pero me gusta el deporte. Voy a estudiar Fisioterapia o Educación Física.''

``Profe, siempre me han gustado las computadoras. Llevo 3 años aprendiendo a programar por mi cuenta. Voy a estudiar Ingeniería en Software.''

Nota el patrón. Todos tenían \textit{años} de exposición a algo que los apasionaba. No estaban adivinando. Estaban decidiendo con evidencia.

\vspace{0.5cm}

Por el otro lado, los hijos de familias de escuelas públicas casi nunca tenían actividades extracurriculares. Llegaban a casa a las 2 de la tarde y pasaban el resto del día frente a una pantalla. Celular. Computadora. Televisión. YouTube. TikTok. Videojuegos.

No era por flojera. Era porque nadie los había expuesto a nada más.

Cuando les preguntaba qué querían estudiar, la respuesta era casi siempre la misma.

``No sé, profe.''

``Lo que pague bien.''

``Lo que mis papás me digan.''

Y estos jóvenes se pierden años valiosos de explorar el mundo, de descubrir qué hay afuera, de conocerse a sí mismos a través de experiencias reales.

No es raro entonces que las familias que toman decisiones intencionales sobre la exposición de sus hijos —no necesariamente las más ricas, sino las más intencionales— tengan hijos que a los 17 años ya saben lo que quieren. O al menos tienen una idea clara de hacia dónde apuntar.

\vspace{0.5cm}

Pero había algo más que distinguía a estas familias. Algo que al principio me pareció menor, pero con el tiempo reconocí como fundamental.

Los hijos de familias estables tomaban sus propias decisiones a partir de los 13, 14, 15 años. Y los papás los dejaban.

Suena obvio, pero la diferencia era dramática.

Muchos de mis alumnos de escuelas privadas eran los que me marcaban directamente para solicitarme clases. No sus papás. Ellos.

``Profe, habla Daniela. Necesito ayuda con cálculo diferencial. ¿Tiene tiempo el jueves a las 5?''

``Profe, soy Roberto. Saqué 6 en el último examen y necesito subir mi promedio. ¿Cuánto cobra por clase? Tengo presupuesto de mi domingo.''

Ellos agendaban la clase. Ellos me pagaban. Cuando llegaba a la casa, ellos me recibían, me ofrecían agua o café, me hacían pasar a la mesa donde íbamos a trabajar. Tenían el material listo. Sabían exactamente qué necesitaban repasar. Se encargaban de todo.

¿Por qué? Porque desde chiquitos los habían entrenado para hacerse cargo de sus propias necesidades.

\vspace{0.5cm}

Por otro lado, en familias sin estas prácticas, el contraste era brutal.

Los papás eran los que me marcaban. Casi siempre en modo de urgencia.

``Maestro, necesito que venga YA. Mi hijo reprobó matemáticas y tiene examen mañana.''

Cuando llegaba, el hijo estaba en su cuarto jugando videojuegos. La mamá tenía que gritarle tres veces para que bajara. Llegaba arrastrando los pies, sin material, sin idea de qué temas venían en el examen. Me miraba con una mezcla de resentimiento y apatía.

Nunca me ofrecían agua. Me sentaba en la mesa del comedor mientras la mamá me explicaba —en voz alta, para que el hijo escuchara— lo ``flojo'' que era su hijo. El joven miraba hacia abajo o hacia su celular, desconectado completamente.

Algunos de estos jóvenes tenían 16, 17 años y no se atrevían a ordenar una pizza por teléfono. Les daba pena hablar con extraños. Les daba ansiedad tomar decisiones simples.

El contraste era devastador. Y todo se reducía a una cosa.

Hábitos.

¿Qué hábitos habían cultivado con sus hijos desde pequeños? ¿A qué los habían expuesto? ¿Qué les habían enseñado sobre responsabilidad personal? ¿Cómo los habían retado progresivamente? ¿Qué tanto conocían el mundo? ¿Qué tanto se conocían a ellos mismos?

\section{Dos tipos de familias y la carrera}

Estas observaciones revelaron algo más profundo. Dos filosofías completamente diferentes sobre la elección de carrera.

\vspace{0.5cm}
\textbf{Familias que confían en la persona}

Las familias económicamente estables —o más bien, las familias intencionales— dejaban que el hijo tomara sus propias decisiones sobre la carrera. Casi siempre. No porque fueran permisivos o desinteresados. Sino porque confiaban en el proceso que habían construido durante años.

``Hija, si quieres estudiar Psicología, adelante. Confío en ti.''

``Hijo, Diseño Gráfico me preocupa por los ingresos, pero si es lo que amas y ya has investigado, te apoyo.''

``Si no funciona, ajustas. Pero la decisión es tuya.''

Claro, había rebeldía ocasional. Como con todo adolescente. Pero era más fácil de navegar cuando había claridad de información y confianza mutua construida desde la infancia.

Y aquí está lo importante. Estas familias no creían en la mentira de que el éxito depende de la carrera que elijas. Ellos sabían algo más profundo. \textbf{De nada sirve que tu hijo sea médico si va a ser un médico miserable, mediocre, que odia su vida.}

Los papás de estas familias querían que sus hijos fueran lo que ellos quisieran ser. Porque confiaban en que habían hecho un buen trabajo formándolos. Confiaban en que su hijo tenía la capacidad de tomar buenas decisiones. Confiaban en que tenía las habilidades y el carácter para tener éxito en cualquier carrera que eligiera.

La carrera era el vehículo. El hijo era el motor.

\vspace{0.5cm}
\textbf{Familias que creen en la carrera mágica}

Por otro lado, en las familias sin estabilidad económica —o sin estas prácticas intencionales— solía operar una filosofía completamente diferente. Una que he visto destruir sueños una y otra vez.

La \textbf{superstición de la carrera mágica}.

Estas familias creían fervientemente que el futuro de su hijo, su éxito, su estabilidad económica, su bienestar general, dependía casi completamente de la carrera que eligiera.

No de quién era como persona. No de su carácter o habilidades. Sino de tres letras en un título universitario.

Entonces, con toda su buena intención, con todo su amor, presionaban.

``Tienes que estudiar Medicina. Es la única forma de que te vaya bien.''

``Ingeniería o nada. Todo lo demás es perder el tiempo.''

``Si no estudias Derecho, vas a terminar de Uber.''

El mensaje implícito era devastador. \textit{No confío en que puedas tener éxito en cualquier campo. Solo puedes salvarte si eliges la carrera ``correcta''.}

Aquí opera la mentira del prestigio social. Elegir una carrera solo por su imagen social —porque ``suena bien'' decir ``mi hijo es doctor'' en reuniones familiares, porque da estatus en la iglesia o en el trabajo— es una de las formas más equivocadas y conflictivas de orientar a un joven.

Muchos renuncian a sus disposiciones naturales más sobresalientes para elegir algo distinto solo por el prestigio que encierra. Pero ese sacrificio personal no reporta gran beneficio ni a ellos ni a su comunidad.

El caso más común que he visto son jóvenes que estudian Derecho sin mayor interés, solo por prestigio social o influencia familiar, en un país donde esa carrera ya está sobresaturada. Muchos terminan en cargos políticos o burocráticos que los dejan al margen del ejercicio real de su profesión. El título de ``doctor'' (en leyes) empieza a fastidiar muy pronto a quien no ve en esa profesión un desafío genuino para sus aptitudes ni puede atribuir valor a tareas que le aburren profundamente.

Y el resultado es predecible. Años de universidad. Deuda estudiantil. Un título enmarcado en la pared. Y una vida de mediocridad infeliz.

\section{El mito de las carreras bien pagadas}

En cada conferencia vocacional hago el mismo ejercicio. Les pregunto a los alumnos cuáles creen que son las tres carreras mejor pagadas. Las respuestas más comunes son Medicina, Ingeniería, Odontología. Luego pregunto cuáles creen que son las peor pagadas. Algunos se ríen, otros dudan, hasta que alguien dice: ``Maestro.'' Risas nerviosas. Después añaden Diseño, Psicología, Artes.

Les cuento historias reales de algunos conocidos. Conozco un médico que trabaja en un hospital público ganando 18,000 pesos al mes después de 11 años de estudio. Vive con sus papás porque no le alcanza para rentar. También conozco un médico con clínica de medicina estética que gana en una semana lo que el primero gana en seis meses. Misma carrera. Dos vidas completamente diferentes.

Conozco ingenieros de universidades privadas manejando Uber porque nadie los contrató. No tiene nada de malo manejar Uber, es un trabajo honesto. Pero no fue para eso que estudiaron cinco años de carrera. Y conozco ingenieros liderando equipos en empresas automotrices con sueldos de seis cifras. Mismo título. Dos realidades opuestas.

Y lo mismo pasa con las carreras ``mal pagadas''. Hay maestros agotados que ganan poco. Y hay docentes que dan clases en universidades privadas, escriben libros, dan conferencias, y tienen muy buenos ingresos. Hay diseñadores gráficos que ganan 8,000 pesos al mes en una imprenta. Y hay diseñadores freelance que facturan 80,000 pesos al mes para clientes internacionales.

Entonces les pregunto: ``Si en las carreras 'buenas' hay gente a la que le va mal, y en las carreras 'malas' hay gente a la que le va increíble... ¿de qué depende realmente el éxito?''

Y ahí lo ven. \textbf{No depende de la carrera. Depende de quién eres como persona.}

La elección vocacional óptima es \textbf{aquella que trae satisfacción personal}. La que te da posibilidad de desarrollar tus mejores aptitudes, de afirmar tus verdaderos intereses, y de no contradecir la estructura básica de tu personalidad.

\section{La vocación no es magia, es proceso}

Antes de cerrar este capítulo, necesito confrontar un mito que paraliza a muchos jóvenes: la idea de que la vocación es un ``llamado misterioso'' que te cae del cielo un día. Que existe una sola carrera ``correcta'' para ti, y si no le ``atinas'', vivirás fracasado el resto de tu vida.

Muchos padres de familia y jóvenes creen esto. Y esa creencia los paraliza. Les genera ansiedad. Los hace sentir que si eligen mal, todo está perdido.

Pero la realidad es otra. Como explica Nuria Cortada de Kohan en su libro sobre orientación vocacional, ``no se nace con una vocación, sino que surge luego de un proceso de madurez y aprendizaje.''\footnote{Cortada de Kohan, N. (2008). \textit{El profesor y la orientación vocacional} (2.ª ed.). Editorial Trillas.} El concepto erróneo de la vocación como algo innato —algo con lo que naces o no naces— ``frustra todo intento de verdadera orientación, porque si la vocación es percibida como algo totalmente innato, orientar vocacionalmente sería una contradicción.''

\textbf{No existe una única carrera perfecta para tus hijos.} Existen varias carreras donde podrían desarrollarse bien. Y la que elijan se irá convirtiendo en ``su vocación'' a medida que dediquen tiempo a ella, la conozcan a fondo, y se especialicen. Siempre y cuando la elijan con base en sus intereses y fortalezas, claro está.

La vocación no aparece de repente. Se va formando a medida que adquirimos mayor experiencia y profundizamos en la realidad de una profesión. Cuando empezamos a conocer a fondo cualquier disciplina, es difícil que no la encuentremos interesante.

Muchos grandes profesionales no empezaron con una vocación perfectamente clara. Tenían intereses, inclinaciones, curiosidades. Y eligieron una carrera que se alineaba con esas inclinaciones, aunque sin certeza absoluta. Pero a medida que profundizaron en ella, esa elección inicial se fue convirtiendo en vocación.

\textbf{La vocación es una forma de expresar nuestra personalidad, intereses y fortalezas frente al mundo del trabajo}. No es un destino escrito en las estrellas. Es un proyecto de vida que se construye con decisiones, experiencias y ajustes constantes.

Como padre, tu trabajo no es ayudar a tu hijo a ``descubrir su vocación misteriosa'', sino a \textbf{construirla}. Y eso requiere trabajo. Para ese trabajo, podemos considerar las siguientes cuatro cosas:

\begin{itemize}
\item Exponerlo a diferentes áreas (deportes, artes, trabajo, servicio)
\item Darle responsabilidades crecientes que desarrollen su carácter
\item Tener conversaciones honestas sobre lo que disfruta y en lo que destaca
\item Darle libertad para explorar sin miedo al fracaso
\end{itemize}

Entonces, aprendemos que la vocación no se encuentra. Se construye. Y tú eres el primer arquitecto de ese proceso en la vida de tus hijos.

\section{Reto del capítulo}

Ahora viene tu parte. Siéntate con tu hijo o hija y tengan una conversación honesta. No un sermón. Una conversación.

\textbf{Pregúntale:}
\begin{itemize}
\item ¿Qué haces en tu tiempo libre que nadie te pide que hagas?
\item ¿En qué actividades pierdes la noción del tiempo?
\item ¿Qué te gustaría probar pero te da miedo o pena intentar?
\end{itemize}

\textbf{Pregúntate:}
\begin{itemize}
\item ¿En qué áreas de la vida he expuesto a mi hijo en los últimos años?
\item ¿He visto a mi hijo destacar en algo sin que nadie se lo pida?
\item ¿Estoy dispuesto a dejar que mi hijo elija una carrera que no me gusta a mí?
\end{itemize}

Al terminar, comprométanse a dos cosas: explorar juntos 2 áreas nuevas en los próximos meses (un curso, un taller, una visita a una empresa) y asignarle una responsabilidad real que maneje solo.

\vspace{0.5cm}

No te obsesiones todavía con QUÉ carrera va a elegir tu hijo. Primero obsesiónate con QUIÉN está siendo.

Si tu hijo desarrolla carácter, habilidades sociales y aprende a resolver problemas, casi cualquier carrera le funcionará. Si no desarrolla eso, ninguna carrera lo salvará.

La vocación no es un tesoro enterrado esperando ser descubierto. Es un músculo que se desarrolla con exposición, práctica, reflexión y decisiones. Y tú eres el entrenador principal.

\vspace{0.5cm}

En el siguiente capítulo vamos a profundizar en esto: qué significa desarrollar carácter y cómo puedes ayudar a tu hijo a construirlo desde casa.

\clearpage
