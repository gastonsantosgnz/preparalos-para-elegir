% Capítulo 5
\chapter{El proceso de elegir}

Ya llegamos hasta aquí. Tu hijo conoce sus fortalezas. Entiende que las carreras existen para resolver problemas reales. Sabe que su éxito depende de conectar lo que es con lo que el mundo necesita.

Ahora viene la parte que muchos padres y muchos hijos temen: \textit{elegir}.

Porque una cosa es entender los principios, y otra muy distinta es sentarse frente a un formulario de inscripción y escribir el nombre de una carrera. Una cosa es hablar de fortalezas y contexto, y otra es comprometerse con una decisión que va a definir los próximos años de tu vida.

Y aquí es donde aparece el miedo.

\vspace{0.5cm}

En mis conferencias de orientación vocacional hay un momento que siempre me sorprende. Ya casi al final, cuando los alumnos están más relajados y rompimos el hielo, les hago una pregunta simple: ``Levanten la mano todos aquellos que han sentido miedo de tomar decisiones.''

La primera vez que hice esta pregunta, no sabía qué esperar. Son adolescentes. ¿De qué van a tener miedo? No han tenido que tomar decisiones importantes en su vida. La mayoría de sus decisiones grandes —qué escuela, qué horario, qué actividades— las han tomado sus papás.

Para mi sorpresa, alrededor del 60 o 70 por ciento de los adolescentes levantaron la mano.

Desde entonces, en cada conferencia repito la misma pregunta. Y siempre, sin excepción, la mayoría levanta la mano. Jóvenes que por fuera se ven seguros, que bromean con sus amigos, que parecen no preocuparse por nada pero por dentro tienen miedo.

\vspace{0.5cm}

Siempre les digo que yo también tuve mucho miedo para tomar decisiones cuando era más joven. Y siendo honesto, a veces todavía lo tengo. El miedo a elegir mal no es exclusivo de los jóvenes. Es parte de ser humano.

Pero también les comparto algo que cambió mi forma de ver las decisiones. Hace años me topé con un libro que me ayudó a entender por qué algunas decisiones son tan difíciles, y sobre todo, qué podemos hacer al respecto. Se llama \textit{Nudge}, que significa ``un pequeño empujón''.\footnote{Thaler, R. y Sunstein, C. (2008). \textit{Nudge: Un pequeño empujón}. Taurus.}

Los autores, Richard Thaler y Cass Sunstein, estudiaron cómo las personas tomamos decisiones en la vida real. No en teoría, sino en la práctica. Y lo que descubrieron explica perfectamente por qué elegir carrera genera tanta ansiedad en los adolescentes.

Lo que aprendí en ese libro es la base de gran parte de este capítulo. Porque si entendemos \textit{por qué} elegir carrera es tan difícil, podemos diseñar un proceso que lo haga más fácil.

\section{Por qué elegir carrera es tan difícil}

Hay una razón por la cual tus hijos sienten miedo de elegir. Y no es porque sean cobardes o inmaduros.

Es porque están tratando de tomar una de las decisiones más importantes de su vida en las peores condiciones posibles para decidir bien. Sí, así es, a los 17 años no todos estamos listos para tomar una de las decisiones más importantes de nuestras vidas.

\vspace{0.5cm}

Según Thaler y Sunstein, las personas tomamos buenas decisiones cuando tenemos tres cosas: \textbf{experiencia previa}, \textbf{buena información} y \textbf{retroalimentación rápida}.

Piénsalo con un ejemplo sencillo. Elegir un sabor de helado es fácil. Tienes experiencia porque ya has probado varios sabores antes. Tienes información porque sabes qué te gusta y qué no. Y la retroalimentación es inmediata: si no te gustó, lo sabes en el primer bocado y la próxima vez eliges otro. Así de sencillo.

Ahora piensa en algo un poco más complejo: elegir un restaurante para cenar. Aquí ya tienes menos experiencia directa, pero puedes leer reseñas, ver fotos del menú, preguntar a amigos que ya fueron. La información está disponible. Y si el restaurante no te gustó, la retroalimentación es rápida: esa misma noche sabes si fue buena decisión o no. No fue tan grave. Aprendiste algo y la próxima vez eliges diferente.

Pero elegir carrera es exactamente lo contrario.

\vspace{0.5cm}

\textbf{No tienen experiencia.} Tus hijos nunca han elegido una carrera antes. Nunca han trabajado como profesionales. Nunca han vivido lo que significa dedicarte a algo ocho horas al día, cinco días a la semana, durante años. Es como pedirle a alguien que nunca ha manejado un carro que elija qué modelo comprar para los próximos diez años.

\textbf{No tienen buena información.} Lo que saben de las carreras viene de películas, redes sociales, comentarios de familiares y estereotipos. Creen que ser abogado es como en las series de televisión, que ser médico es como en \textit{Grey's Anatomy}, que ser empresario es como lo pintan los influencers. No tienen idea de cómo es el día a día real de esas profesiones.

\textbf{La retroalimentación es lentísima.} Esta es la por parte. Tus hijos solo sabrán si tomaron una buena decisión hasta años después, cuando ya invirtieron tiempo, dinero y esfuerzo. No es como el helado que pruebas y en un minuto sabes si te gustó. Aquí pueden pasar dos, tres, cuatro años antes de que realmente entiendan si eligieron bien. Y para entonces, el costo de cambiar es enorme.

\vspace{0.5cm}

Por eso elegir carrera genera tanta ansiedad. No es un problema de actitud. Es un problema estructural.

Y si tus hijos están confundidos, no es por falta de capacidad o de ganas. Es porque están en desventaja. Nadie les ha dado las herramientas para decidir bien.

Pero hay algo más que complica las cosas para esta generación en particular.

\subsection{Informados pero deformados}

Uno de los principales problemas de los adolescentes de hoy es que no saben para qué son buenos. No han vivido lo suficiente. No han tenido experiencias reales que les muestren sus fortalezas.

Si le preguntas a tu hijo qué le gusta hacer, probablemente te conteste que le gusta estar en redes sociales, jugar videojuegos o ver videos en TikTok. Porque eso es a lo que está expuesto todo el día.

Y si le preguntas qué quiere ser de grande, no es raro escuchar respuestas como YouTuber, TikToker, streamer, creador de contenido. Porque eso es lo que ve. Eso es lo que conoce.

\vspace{0.5cm}

Pero hay un problema. Eso que desea no es algo que haya experimentado en la realidad. Lo ha experimentado a través de una pantalla, a través de la vida editada de otras personas, no de la vida real.

\textbf{Están informados, pero no están formados.}

Esta es la generación más informada y menos formada de toda la historia. Tienen acceso a más datos que cualquier generación anterior, pero no han tenido suficientes experiencias que formen su carácter, sus habilidades, sus fortalezas. Han observado el mundo desde afuera, pero no lo han vivido.

\vspace{0.5cm}

Por eso tantos adolescentes se sienten confundidos cuando llega el momento de elegir. No es que no quieran decidir. Es que no tienen la materia prima para hacerlo bien. No tienen experiencia real. No tienen información confiable. Y no tienen forma de saber si eligieron bien o no hasta que ya es demasiado tarde.

Y aquí es donde entramos nosotros como padres. Porque aunque no podemos tomar la decisión por ellos, sí podemos ayudarles a conseguir lo que les falta. Podemos darles herramientas, experiencias y perspectiva. Podemos acompañarlos en el proceso.

\section{Lo nuevo siempre da miedo}

Elegir algo nuevo siempre puede dar miedo. Y eso está bien.

Lo nuevo representa incertidumbre. Y la incertidumbre genera ansiedad. Es parte de ser humano. A nadie le gusta sentir que está dando un salto sin saber dónde va a caer.

El problema no es el miedo. El problema es quedarse paralizado por él.

\vspace{0.5cm}

Piensa en tu propia vida. ¿Recuerdas cuando elegiste tu carrera? ¿Cuando aceptaste tu primer trabajo? ¿Cuando decidiste casarte o tener hijos? Seguramente había miedo. Pero actuaste de todos modos. Y aquí estás.

Ahora tu hijo está en ese mismo punto. Y esta es una muy buena oportunidad para pedirle que sea valiente.

Han crecido toda su vida bajo nuestra protección. Todas sus decisiones importantes —qué escuela, qué horario, qué actividades— las hemos tomado nosotros. Los hemos cuidado, supervisado, guiado. Y eso estuvo bien. Era nuestro trabajo.

Pero ahora llega el momento donde ellos deben empezar a tomar sus propias decisiones. Y no siempre va a ser fácil. No siempre la respuesta va a ser evidente. Van a necesitar ser valientes para tomar el riesgo, para escoger, para hacer lo necesario para que esta decisión sea lo mejor posible.

Y en esa valentía existe también la posibilidad de equivocarse. No hay forma de evitarlo completamente. Pero eso también es parte del proceso. Aprender a manejar el error es una habilidad que les va a servir toda la vida.

\vspace{0.5cm}

Ahora bien, valentía no significa lanzarse a ciegas. Valentía significa actuar a pesar del miedo, pero con preparación.

\textbf{Y aquí está la buena noticia. Con la estrategia correcta, tomar decisiones difíciles puede no ser tan difícil.}

\section{Tres pasos para elegir bien}

Si el problema es falta de experiencia, falta de información y falta de retroalimentación rápida, entonces la solución es clara. Necesitamos darles esas tres cosas antes de que tomen la decisión.

No después. Antes.

\vspace{0.5cm}

En el libro \textit{Nudge} que mencioné al principio, los autores hablan de cómo diseñar mejores procesos de decisión. La idea central es simple. En lugar de esperar que la gente tome buenas decisiones por arte de magia, hay que crear las condiciones para que sea más fácil decidir bien.

Eso es exactamente lo que vamos a hacer aquí.

Hay tres pasos que pueden transformar la forma en que tu hijo elige carrera. No son complicados, pero requieren intención. No suceden solos. Alguien tiene que provocarlos.

Ese alguien eres tú.

\subsection{Paso 1: Buscar información real}

La mayoría de los adolescentes eligen carrera basándose en imágenes mentales, no en información real.

Piensan: ``Quiero ser médico porque salvan vidas y ganan bien.'' Pero no saben que un médico trabaja guardias de 24 horas, que los primeros años gana poco, que hay mucha burocracia hospitalaria, que el desgaste emocional es fuerte.

Piensan: ``Quiero estudiar Derecho porque es una carrera respetada.'' Pero no saben cómo es realmente el día a día de un abogado, qué tipo de problemas resuelve, cuántas horas trabaja, qué tan saturado está el mercado.

\textbf{La imagen que tienen de las carreras casi nunca coincide con la realidad.}

\vspace{0.5cm}

Por eso el primer paso es investigar. Pero no investigar superficialmente. Investigar de verdad.

Para cada carrera que tu hijo esté considerando, debería poder responder estas preguntas:

\textit{Sobre la carrera académica:}
\begin{itemize}
\item ¿Qué materias se estudian cada semestre?
\item ¿Cuántos años dura realmente? ¿Hay que hacer servicio social, prácticas, tesis?
\item ¿Cuál es la tasa de deserción? ¿Qué porcentaje de estudiantes la termina?
\item ¿Cuánto cuesta en total, incluyendo materiales, transporte, tiempo sin trabajar?
\end{itemize}

\textit{Sobre el mercado laboral:}
\begin{itemize}
\item ¿Qué trabajos específicos puede conseguir un egresado?
\item ¿Cuánto gana alguien recién egresado? ¿Y alguien con 10 años de experiencia?
\item ¿Hay demanda real de esta profesión en mi ciudad o región?
\item ¿Se requiere posgrado para tener éxito?
\end{itemize}

\textit{Sobre la realidad del trabajo:}
\begin{itemize}
\item ¿Cómo es un día típico de trabajo en esta profesión?
\item ¿Cuáles son los mejores aspectos? ¿Y los peores?
\item ¿Qué problemas resuelve esta profesión todos los días?
\item ¿Hay flexibilidad de horario? ¿Posibilidad de trabajo remoto?
\end{itemize}

\vspace{0.5cm}

Si tu hijo no puede responder estas preguntas, no está listo para elegir esa carrera. No porque sea incapaz, sino porque no tiene la información necesaria.

Y eso se puede arreglar. Solo necesita investigar.

\subsection{Paso 2: Buscar experiencias de otros}

La información escrita es útil, pero tiene límites. Los números y los datos no te cuentan cómo se siente realmente trabajar en algo.

Para eso necesitas hablar con personas que ya lo viven.

\vspace{0.5cm}

Le llamo a esto ``las cinco conversaciones''. Tu hijo debería entrevistar a por lo menos cinco profesionales que actualmente trabajan en las carreras que está considerando.

¿Por qué cinco? Porque una sola experiencia puede ser atípica. Tal vez esa persona tuvo mucha suerte, o mucha mala suerte. Hablar con cinco o más personas te da una imagen más realista y balanceada.

\textbf{¿A quién entrevistar?}

\begin{itemize}
\item Dos personas con menos de cinco años de egresados, para ver la realidad inicial
\item Dos personas con diez o más años de experiencia, para ver el desarrollo de carrera
\item Una persona que cambió de carrera o que se arrepintió, para ver la otra cara
\end{itemize}

\textbf{¿Dónde encontrarlos?}

\begin{itemize}
\item Contactos familiares y de amigos
\item LinkedIn, donde muchos profesionales están dispuestos a ayudar a jóvenes
\item Eventos universitarios, ferias de empleo
\item Oficinas, hospitales, despachos, pidiendo treinta minutos de su tiempo
\end{itemize}

\vspace{0.5cm}

\textbf{Preguntas clave para la entrevista:}

\begin{enumerate}
\item ¿Por qué elegiste esta carrera?
\item ¿Qué te sorprendió más de la realidad profesional comparado con lo que esperabas?
\item Describe un día típico de tu trabajo.
\item ¿Qué es lo que más disfrutas de tu trabajo?
\item ¿Qué es lo más difícil o frustrante?
\item Si pudieras volver atrás, ¿elegirías la misma carrera? ¿Por qué sí o no?
\item ¿Qué consejo le darías a alguien que está considerando esta carrera?
\item ¿Qué habilidades o fortalezas son esenciales para tener éxito en este campo?
\end{enumerate}

\vspace{0.5cm}

Este ejercicio hace algo muy poderoso: destruye las imágenes falsas.

Tu hijo va a descubrir aspectos de la profesión que nunca aparecen en los folletos universitarios. Va a escuchar historias reales, con frustraciones reales y satisfacciones reales. Y va a poder imaginar si él realmente encaja ahí.

\subsection{Paso 3: Buscar experiencias propias}

La información y las conversaciones son valiosas. Pero nada sustituye la experiencia propia.

La verdad es que hay cosas que tus hijos no van a saber hasta que las experimenten. Incluso aunque lo sospechen, no hay mejor forma de saber algo que cuando te consta.

Yo no aprendí que era bueno para las ventas hasta que entré a un trabajo donde tenía que vender. No aprendí que era bueno para el servicio al cliente hasta que entré a un trabajo donde tenía que dar servicio al cliente. Y no aprendí que era malo para organizar archivos y administrar hasta que tuve que hacerlo en un trabajo real.

\vspace{0.5cm}

Por eso el tercer paso es probar en pequeño antes de comprometerse en grande.

\textbf{Formas de experimentar:}

\textit{Shadowing.} Acompañar a un profesional durante uno o dos días de trabajo. Observar todo: reuniones, tareas, interacciones, decisiones. Ver cómo es realmente el ritmo.

\textit{Prácticas cortas o trabajo de verano.} Conseguir un trabajo temporal o práctica en el área de interés. Si considera Medicina, voluntariado en hospital. Si considera Ingeniería, práctica en empresa de manufactura. Si considera Diseño, proyectos freelance pequeños.

\textit{Cursos introductorios.} Tomar un curso básico sobre la materia principal de la carrera. Si considera Programación, un curso de Python por cuatro semanas. Si considera Psicología, ``Introducción a la Psicología'' en Coursera. Si odia el curso básico, probablemente odiará la carrera.

\textit{Proyectos personales.} Crear algo relacionado con la carrera como experimento. Si considera Comunicación, iniciar un blog o podcast por tres meses. Si considera Negocios, vender algo en línea. Si considera Arte, producir una serie de obras.

\vspace{0.5cm}

El objetivo es convertir una decisión abstracta en una decisión con evidencia.

Al final de este paso, tu hijo debería poder decir:

\begin{itemize}
\item ``He probado dos o tres de mis opciones en pequeño.''
\item ``Sé cómo se siente hacer este tipo de trabajo.''
\item ``Tengo evidencia de que puedo y me gusta hacer esto.''
\item ``Elegir ya no es adivinar. Es decidir con información.''
\end{itemize}

\section{El default reflexivo}

Ahora quiero hablar de algo que muchos padres me preguntan en mis conferencias.

``Maestro, ¿está mal que mi hijo estudie lo mismo que yo estudié? Tengo un despacho de contadores y me gustaría que mi hijo lo continuara. ¿Eso está mal?''

O también: ``Mi familia siempre ha sido de médicos. Mi papá, mi abuelo, mis tíos. ¿Está mal que mi hijo siga esa tradición?''

Mi respuesta siempre los sorprende: no necesariamente.

\vspace{0.5cm}

No creo que seguir el camino de los padres sea automáticamente malo. De hecho, creo que puede ser una muy buena alternativa.

Piénsalo un momento. Como papá, ya creaste toda una estructura alrededor de tu profesión: un estilo de vida, relaciones sociales, contactos profesionales, clientes, reputación. Tu hijo puede aprovechar todo eso si decide estudiar lo mismo que tú, dedicarse a lo mismo que tú te dedicas.

He conocido muchas personas que se dedican a lo que hicieron sus padres. Y por esa familiaridad, por esas habilidades desarrolladas casi sin darse cuenta a través de los años —escuchando conversaciones en la cena, acompañando a papá al trabajo, viendo cómo resuelve problemas— tienden a ser suficientemente buenos para desarrollar una buena vida alrededor de eso.

No tienen que empezar de cero. Ya tienen ventaja.

\textbf{Hay cierta seguridad en el camino conocido.}

\vspace{0.5cm}

Sin embargo, tal vez eso no es suficiente para alguien que realmente quiere sacar su mejor potencial.

Hay una historia que me gusta mucho y que ilustra esto perfectamente:

\begin{quote}
Un hombre murió y fue al cielo. Al llegar, se encontró con San Pedro en las Puertas Celestiales.

El hombre le dijo a San Pedro que durante su vida había estudiado muchas biografías de militares y generales. Siempre había tenido curiosidad: quería saber, a ciencia cierta, quién fue el mejor general que haya existido.

San Pedro respondió que podía ayudarle, y lo condujo a otra parte del cielo, una sala donde le presentaron a un hombre sencillo, vestido de manera humilde, sin uniforme, sin medallas, con manos comunes.

El hombre, sorprendido, exclamó: ``Ese no puede ser. Yo lo conocí cuando vivía en la tierra. Él no era un general. Era un simple zapatero.''

San Pedro lo miró con calma y dijo: ``Tienes razón. Pero si ese hombre hubiera sido general, si le hubieran dado la oportunidad, si se le hubiera puesto a entrenar, si le hubieran confiado un ejército, habría sido el mejor general de todos los tiempos.''
\end{quote}

Esa es la conclusión: muchas personas ordinarias tienen dentro el talento para ser extraordinarias. El problema es que nunca se les permite ser lo que podrían ser.

\vspace{0.5cm}

Tal vez ese zapatero de la historia era zapatero porque su papá fue zapatero, y el papá de su papá también. Y no tiene nada de malo. Tal vez vivió una buena vida, con familia, hijos, nietos. Pero nunca logró su mejor potencial, porque nadie lo expuso a aquello que pudo haber sido.

\subsection{La diferencia está en la reflexión}

En la película \textit{You've Got Mail} hay una escena que me gusta mucho. Es un monólogo interno de Kathleen Kelly, la protagonista:

\begin{quote}
``A veces me pregunto sobre mi vida. Vivo una vida pequeña, valiosa, pero pequeña. Y a veces me pregunto, ¿lo hago porque me gusta, o porque no he sido lo bastante valiente?''
\end{quote}

Kathleen Kelly era encargada de la librería que le heredó su mamá. Era muy feliz, era plena encargándose de esa librería. Desde chiquita le ayudaba a su mamá a organizar los libros, atender clientes, limpiar, cobrar.

Pero llega un momento en su vida donde tiene que cerrar la librería. Y es entonces cuando le da un golpe de realidad. No había conocido nada más en la vida. Y se pregunta: ¿estoy haciendo esto porque me gusta, porque me tocó, o porque no he sido valiente?

\vspace{0.5cm}

Creo que esta condición del \textbf{default reflexivo} es cuando nosotros tenemos la oportunidad de decir que no.

Cuando tenemos la oportunidad de renunciar a esas oportunidades fáciles de la vida, pero decidimos no hacerlo. Cuando tomamos la decisión de sí tomar este camino, de sí elegir esa carrera, aunque haya sido la misma que escogieron nuestros padres.

\textbf{El default reflexivo es cuando eliges el camino conocido después de considerar las alternativas.} No por inercia. No por miedo. Por decisión consciente.

Y creo que como padres de familia, nosotros tenemos que darles esa libertad a nuestros hijos.

Al final de cuentas, libertad es tener al menos dos opciones. Si solamente tienen una opción, no es libertad.

\section{Los tests vocacionales}

Tarde o temprano, alguien le va a sugerir a tu hijo que haga un test vocacional. Tal vez en la escuela, tal vez un orientador, tal vez un familiar bien intencionado.

Y tú te vas a preguntar: ¿sirven? ¿No sirven? ¿Vale la pena hacerlos?

\vspace{0.5cm}

Mi posición es esta. Los tests vocacionales son útiles, pero no son mágicos. Son una herramienta más en el proceso de decisión. Una herramienta, pero no la única.

\textbf{No construyes una casa solo con un martillo.} Y no eliges carrera solo con un test.

El problema es que muchas familias tratan el test vocacional como si fuera un oráculo. Como si al contestar cien preguntas, una computadora pudiera decirte exactamente qué estudiar. Y eso genera expectativas irreales que terminan en frustración.

\subsection{Qué son y para qué sirven}

Un test vocacional es una herramienta que busca ordenar información sobre una persona para ayudarle a explorar opciones académicas y laborales.

No ``descubre'' una vocación escondida. Más bien mide patrones que suelen relacionarse con preferencias: qué actividades disfrutas, qué se te facilita aprender, cómo tiendes a actuar, qué consideras importante.

Su propósito es práctico: mejorar decisiones en contextos donde elegir ``a ojo'' sale caro.

\textbf{Un buen test no te da una sentencia. Te da hipótesis útiles para explorar.}

\vspace{0.5cm}

Hay diferentes tipos de tests:

\textit{Tests de intereses} preguntan qué actividades te atraen y te devuelven familias de ocupaciones afines. Útiles para explorar, no para predecir éxito.

\textit{Tests de aptitudes} estiman qué tan fácil podrías aprender ciertas tareas: verbales, numéricas, espaciales.

\textit{Tests de personalidad} usan rasgos para sugerir entornos donde la persona suele funcionar mejor.

\textit{Tests de valores} buscan qué motivaciones te mueven: servicio, dinero, conocimiento, estética, poder.

\subsection{Lo que los tests sí hacen bien}

Los tests tienen ventajas reales cuando se usan correctamente.

\textbf{Aceleran la conversación correcta.} Muchos adolescentes no tienen lenguaje para describirse. Un test convierte intuiciones en categorías discutibles. Eso ya es ganancia.

\textbf{Reducen el ruido social.} Cuando el entorno mete presión, un test puede funcionar como espejo para separar deseo real de expectativas ajenas.

\textbf{Sirven como mapa para investigar.} Su valor está en lo que haces después. Te dan un punto de partida para los tres pasos que describimos antes.

\textbf{Ayudan a perfiles indecisos.} Si tu hijo tiene veinte ideas y cero criterio para filtrar, un test puede crear un top tres para enfocar la investigación.

\subsection{Lo que los tests no hacen}

Ahora viene la parte que muchos no quieren escuchar.

\textbf{Los tests no predicen éxito.} Tener interés alto en algo no significa que tendrás disciplina, tolerancia al fracaso, o habilidades para sobresalir. El éxito depende del carácter que vimos en el Capítulo 2, no de un resultado de test.

\textbf{Los tests no capturan el contexto real.} Economía local, oportunidades, red de contactos, restricciones familiares, dinero, movilidad. Un adolescente puede ``salir'' Investigador y vivir en un lugar donde esa ruta requiere mudarse.

\textbf{Los tests no miden motivación sostenida.} Contestar ``me gusta'' no es lo mismo que sostener cuatro años de carrera y luego ocho horas diarias de trabajo.

\textbf{Los tests tienen un porcentaje de error.} Tal vez por la eficiencia del mismo test, o por las respuestas sesgadas del alumno que contesta lo que ``suena bien'' en lugar de lo que realmente es.

\vspace{0.5cm}

\textbf{Lo que los tests NO te dicen:}

\begin{itemize}
\item Si tu hijo tendrá disciplina, resiliencia y buen carácter
\item Si el mercado local o global favorecerá esa ruta en cinco o diez años
\item Si odiará el día a día real del trabajo
\item Si esa elección encaja con valores familiares y estilo de vida
\end{itemize}

\subsection{Cómo usar los tests correctamente}

Mi recomendación es esta. Hacer al menos tres tests vocacionales. Cinco sería lo ideal.

¿Por qué varios? Porque cada test hace diferentes estilos de preguntas. En diferentes contextos, con diferentes palabras, el alumno puede pensar o responder diferente.

A partir de tres, cuatro, cinco tests, ya podemos empezar a recibir información donde podemos encontrar algún patrón.

Por ejemplo, en el primer test tal vez salió que se le daba la ingeniería. Pero en el segundo y tercero también apareció economía y marketing. Entonces ingeniería también, pero en menos proporción. El primero no fue la solución completa, pero después del segundo, tercero y cuarto ya se ve el panorama completo de los intereses.

\vspace{0.5cm}

\textbf{Los resultados son un punto de partida, no un punto de llegada.}

No toman la decisión por nosotros. Deben ser el inicio para empezar a investigar esas áreas y carreras usando los tres pasos que ya vimos.

Los resultados nos pueden dar un norte, indicar por dónde empezar a investigar. Nos pueden orientar para saber qué trabajo puede tomar tu hijo de medio tiempo, en fines de semana, en vacaciones de verano o de invierno, para que él empiece a experimentar.

\vspace{0.5cm}

\textbf{Fórmula útil:}

Test (hipótesis) + Fortalezas (Capítulo 3) + Experiencia real (tres pasos) + Contexto (Capítulo 4) = decisión más inteligente.

\section{Criterios para la decisión final}

Llegó el momento que todos esperaban. Tu hijo investigó, habló con profesionales, experimentó en pequeño, hizo sus tests. Tiene dos o tres opciones finalistas que le gustan.

¿Y ahora qué? ¿Cómo elige entre ellas?

\vspace{0.5cm}

Aquí es donde muchas familias se paralizan. Sienten que ya hicieron todo el trabajo, pero la decisión final sigue siendo difícil. Y tiene sentido. Si las opciones fueran obvias, no habrían llegado hasta aquí.

Lo que necesitan es un marco de evaluación. Una forma de comparar las opciones usando criterios claros, no solo intuición.

Aquí hay cinco criterios que pueden ayudar:

\vspace{0.5cm}

\textbf{1. Alineación con fortalezas}

¿Esta carrera utiliza los talentos naturales más fuertes de tu hijo? ¿Tiene evidencia de que puede ser muy bueno en esto?

Recuerda lo que vimos en el Capítulo 3: Talento × Inversión = Fortaleza. Una carrera bien elegida potencia lo que ya tiene, no intenta arreglar lo que le falta.

\textbf{2. Disfrute sostenido}

¿Disfruta haciendo este tipo de trabajo, no solo la idea de la carrera? ¿Puede verse haciendo esto durante los próximos diez o veinte años?

Hay una diferencia enorme entre ``me gusta la idea de ser médico'' y ``me gusta estudiar biología, anatomía, y atender pacientes''. La primera es fantasía. La segunda es evidencia.

\textbf{3. Viabilidad económica}

¿Hay demanda real de esta profesión? ¿Los ingresos potenciales justifican la inversión en educación? ¿Puede costear la carrera sin endeudarse excesivamente?

Esto no significa elegir ``la carrera que más paga''. Significa ser realista sobre el contexto, como vimos en el Capítulo 4.

\textbf{4. Impacto y propósito}

¿Esta profesión resuelve problemas que le importan? ¿Siente que estaría contribuyendo algo valioso al mundo?

Recuerda: las carreras existen para resolver problemas reales. Una carrera con propósito es más fácil de sostener que una sin él.

\textbf{5. Flexibilidad y opciones futuras}

Si cambia de opinión, ¿esta carrera le da habilidades transferibles? ¿Hay múltiples caminos laborales dentro de esta profesión?

El mundo cambia. Las carreras cambian. Una buena elección deja puertas abiertas, no las cierra.

\vspace{0.5cm}

\textbf{No existe la elección perfecta.} Existe la elección bien informada que tu hijo puede convertir en una gran decisión a través de su esfuerzo y dedicación.

\section{Reto del capítulo}

Este reto es diferente a los anteriores. No es un ejercicio de una tarde. Es un proceso que puede tomar semanas o incluso meses.

Pero no dejes que eso te desanime. Lo que estás a punto de hacer con tu hijo puede cambiar la trayectoria de su vida. Vale cada minuto invertido.

\vspace{0.5cm}

\textbf{Paso 1: Filtrar opciones}

Si tu hijo todavía no tiene claridad, usen los filtros del Capítulo 4:

\begin{itemize}
\item ¿En qué fortalezas tiene evidencia real? (Capítulo 3)
\item ¿Qué problemas le importa resolver?
\item ¿Qué opciones son viables en su contexto?
\item ¿Qué opciones cumplen los cuatro círculos del ikigai?
\end{itemize}

El resultado debería ser tres a cinco carreras finalistas para investigar a fondo.

\vspace{0.5cm}

\textbf{Paso 2: Investigar información real}

Para cada carrera finalista, que tu hijo investigue y responda:

\begin{itemize}
\item ¿Qué materias se estudian?
\item ¿Cuánto dura y cuánto cuesta?
\item ¿Qué trabajos puede conseguir un egresado?
\item ¿Cuánto gana alguien con esta carrera en su región?
\item ¿Cómo es un día típico de trabajo?
\end{itemize}

\vspace{0.5cm}

\textbf{Paso 3: Las cinco conversaciones}

Que tu hijo entreviste a por lo menos cinco profesionales de las carreras que está considerando. Pueden ser contactos familiares, conocidos, o personas que contacte por LinkedIn.

Las preguntas clave están en este capítulo. El objetivo es escuchar historias reales, no teoría.

\vspace{0.5cm}

\textbf{Paso 4: Experimentar en pequeño}

Antes de decidir, que tu hijo pruebe al menos dos de sus opciones finalistas:

\begin{itemize}
\item Un día de shadowing con un profesional
\item Un curso introductorio en línea
\item Un proyecto personal relacionado
\item Un trabajo de verano o práctica corta
\end{itemize}

\vspace{0.5cm}

\textbf{Paso 5: Los tests vocacionales}

Que tu hijo tome tres a cinco tests vocacionales de fuentes confiables. Comparen los resultados:

\begin{itemize}
\item ¿Qué patrones aparecen en varios tests?
\item ¿Los resultados coinciden con lo que ya sabían?
\item ¿Hay sorpresas que vale la pena explorar?
\end{itemize}

Usen los resultados como hipótesis para investigar, no como veredicto final.

\vspace{0.5cm}

\textbf{Paso 6: La decisión}

Después de completar los pasos anteriores, siéntense a evaluar las opciones usando los cinco criterios de este capítulo:

\begin{enumerate}
\item Alineación con fortalezas
\item Disfrute sostenido
\item Viabilidad económica
\item Impacto y propósito
\item Flexibilidad futura
\end{enumerate}

\vspace{0.5cm}

\vspace{0.5cm}

\textbf{Recuerda:} La mejor decisión no es la perfecta. Es la reflexiva, informada y valiente.

Tu hijo no necesita adivinar su futuro. Necesita construirlo con información, experiencia y reflexión. Y tú estás ahí para acompañarlo en el proceso.

No para decidir por él. Para caminar junto a él mientras decide.

\vspace{0.5cm}

Ahora bien, ¿qué pasa después de que elige? ¿Qué pasa cuando ya está en la universidad y las cosas no salen como esperaba? ¿Qué haces si quiere cambiar de carrera? ¿O si quiere dejarla?

En el siguiente capítulo vamos a hablar de algo que muchos padres evitan pensar: qué hacer cuando ya eligió. Cómo acompañarlo en los primeros meses. Cómo detectar señales de alerta. Y cómo manejar los escenarios difíciles —porque van a aparecer— sin destruir la relación ni el futuro de tu hijo.

\clearpage
