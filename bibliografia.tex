% Bibliografía

\begin{thebibliography}{99}

\bibitem{birch2007}
Birch, E.R., \& Miller, P.W. (2007). The influence of type of high school attended on university performance. \textit{Australian Economic Papers}, 46(1), 1-17.

\bibitem{cloud2006}
Cloud, H. (2006). \textit{Integrity: The Courage to Meet the Demands of Reality}. HarperBusiness.

\bibitem{cortada2008}
Cortada de Kohan, N. (2008). \textit{El profesor y la orientación vocacional} (2.ª ed.). Editorial Trillas. ISBN 978-968-24-7992-2.

\bibitem{garcia2017}
García, H., \& Miralles, F. (2017). \textit{Ikigai: The Japanese Secret to a Long and Happy Life}. Penguin Books.

\bibitem{hammitt2024}
Hammitt, H.M. (2024). \textit{Democratizing the Gap Year Option at the Community College Level}. Bradley University.

\bibitem{rabie2016}
Rabie, S., \& Naidoo, A.V. (2016). The value of the gap year in the facilitation of career adaptability. \textit{South African Journal of Higher Education}, 30(3), 163-177.

\bibitem{rath2007}
Rath, T. (2007). \textit{StrengthsFinder 2.0}. Gallup Press. ISBN 978-1595620156.

\bibitem{silva2021}
Silva, D. (2021). \textit{The Impact of Gap Years on Personal \& Career Related Growth}. Nova School of Business and Economics.

\bibitem{thaler2008}
Thaler, R., \& Sunstein, C. (2008). \textit{Nudge: Un pequeño empujón}. Taurus.

\end{thebibliography}









